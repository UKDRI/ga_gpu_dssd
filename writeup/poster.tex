% Created 2025-12-04 Thu 13:06
% Intended LaTeX compiler: pdflatex
\documentclass[a0paper]{article}
\usepackage[utf8]{inputenc}
\usepackage[T1]{fontenc}
\usepackage{graphicx}
\usepackage{longtable}
\usepackage{wrapfig}
\usepackage{rotating}
\usepackage[normalem]{ulem}
\usepackage{amsmath}
\usepackage{amssymb}
\usepackage{capt-of}
\usepackage{hyperref}
\usepackage{listings}
\usepackage[margin=1.5cm, top=1.5cm, bottom=1.5cm]{geometry}
\usepackage{multicol}
\usepackage{graphicx}
\usepackage{xcolor}
\usepackage{float}
\usepackage{helvet}
\renewcommand{\familydefault}{\sfdefault}
\setlength{\parindent}{0pt}
\setlength{\parskip}{0.2em} % Increased slightly to fill space
\usepackage{titlesec}
\titlespacing*{\section}{0pt}{0.4cm}{0.2cm}
\titlespacing*{\subsection}{0pt}{0.2cm}{0.1cm}
\titleformat{\section}{\fontsize{50}{60}\selectfont\bfseries\color{blue}}{}{0em}{}
\titleformat{\subsection}{\fontsize{32}{40}\selectfont\bfseries\color{darkgray}}{}{0em}{}
\usepackage{caption}
\DeclareCaptionFont{posterfont}{\fontsize{28}{34}\selectfont}
\captionsetup{font={posterfont},labelfont={bf,color=blue},skip=12pt}
\setlength{\columnsep}{2.5cm}
\date{}
\title{Poster Content}
\hypersetup{
 pdfauthor={Sam Neaves},
 pdftitle={Poster Content},
 pdfkeywords={},
 pdfsubject={},
 pdfcreator={Emacs 29.3 (Org mode 9.6.15)}, 
 pdflang={English}}
\begin{document}

% START TWO COLUMN LAYOUT
\begin{multicols}{2}
\fontsize{28}{34}\selectfont % Slightly larger body text since you have room

\section*{1. The Challenge: The "Mixed Bag"}
\label{sec:org3b749d1}
Standard clustering groups cells by Global Transcriptomic Identity,
forcing cells into mutually exclusive groups. By ignoring the disease
label of individual cells, this approach obscures specific functional
failure modes that cut across neuronal subtypes.


\textbf{\textbf{Objective:}} To use GPU-accelerated Subgroup Discovery to find \textbf{\textbf{robust combinatorial rules}} that define these hidden states without prior biological assumptions.

\section*{2. The Solution: GPU-Accelerated Discovery}
\label{sec:orge21e088}
To ensure robust biological signals, single-cell expression data was
\textbf{\textbf{Z-scored}}, \textbf{\textbf{donor-regressed}}, and \textbf{\textbf{discretized}}. To tackle
complexity, we developed a custom \textbf{\textbf{Parallel Genetic Algorithm}} to
perform Diverse Subgroup Set Discovery (DSSD) on GPU architecture:

\begin{itemize}
\item \textbf{\textbf{Glass-Box AI:}} Evolves human-readable logical rules (e.g., \emph{IF Gene A AND NOT Gene B}).
\item \textbf{\textbf{Set Optimization:}} Optimizes a \textbf{portfolio} of rules simultaneously, penalizing overlap.
\item \textbf{\textbf{High-Throughput:}} Leveraged \textbf{\textbf{\textasciitilde{}10,000 GPU hours}} on the \textbf{\textbf{DAWN AI Supercomputer}} to perform massive parallel permutation testing, ensuring rigorous statistical validation.
\end{itemize}

% --- FIGURE 1: METHOD & VALIDATION (Side-by-Side) ---
\begin{center}
    \begin{minipage}{0.58\linewidth}
        % The Method Diagram
        \includegraphics[width=\linewidth,height=12cm,keepaspectratio]{./images/GPU_GA.pdf}
        \captionof{figure}{\textbf{The Discovery Engine.} Parallel evolution of rule sets.}
    \end{minipage}
    \hfill
    \begin{minipage}{0.38\linewidth}
        % The Validation Histogram
        \includegraphics[width=\linewidth,height=10cm,keepaspectratio]{./images/fitness_histogram.png}
        \captionof{figure}{\textbf{Validation.} Signal vs Noise ($p < 0.001$).}
    \end{minipage}
\end{center}

\section*{3. Discovery: Three Distinct Failure Modes}
\label{sec:org9616906}
The DSSD framework identified three robust, distinct \textbf{\textbf{"Functional States"}} within the heterogeneous PD population.


% --- FIGURE 2: RULES & OVERLAP ---
\begin{center}
    \begin{minipage}{0.58\linewidth}
        \includegraphics[width=\linewidth,height=14cm,keepaspectratio]{./images/subgroup_analysis.png}
        \captionof{figure}{Discovered Rules.}
    \end{minipage}
    \hfill
    \begin{minipage}{0.38\linewidth}
        \includegraphics[width=\linewidth,height=14cm,keepaspectratio]{./images/subgroup_overlap_venn.png}
        \captionof{figure}{\textbf{Distinct States.} Minimal overlap.}
    \end{minipage}
\end{center}

\section*{4. Biological Characterization}
\label{sec:orgd7e3730}
Functional enrichment analysis (Enrichr) reveals that each subgroup is
charaterized by distinct pathological mechanism, supporting the "Orthogonal Failure Mode" hypothesis.

\begin{itemize}
\item \textbf{\textbf{Rule 1 (Metabolic):}} Characterized by a \textbf{\textbf{bioenergetic collapse}}, indicated by extreme enrichment for \emph{Mitochondrial Myopathy} and \emph{Leigh Disease} signatures (\(P < 10^{-18}\)).
\item \textbf{\textbf{Rule 2 (Iron/Stress):}} Defined by the dysregulation of \textbf{\textbf{Iron Homeostasis}} and \emph{Secretory Granules}, pointing to specific lysosomal and vesicular toxicity.
\item \textbf{\textbf{Rule 3 (Regeneration):}} Displays a \textbf{\textbf{"Frustrated Regeneration"}} phenotype. The upregulation of acute injury factors (\emph{EGR1}) and \textbf{\textbf{Axon Guidance}} machinery suggests an active attempt at structural repair.
\end{itemize}

% --- FIGURE 3: THE DOT PLOT ---
\begin{center}
    \includegraphics[width=0.95\linewidth,height=18cm,keepaspectratio]{./images/PD_Enrichr_DotPlot_From_Files.png}
    \captionof{figure}{\textbf{Deconvolution of Pathology.} Analysis against GO, Jensen Diseases, and TF databases reveals three non-overlapping mechanisms.}
\end{center}

\section*{5. Impact: Functional States vs. Identity}
\label{sec:orgfbf7844}
Standard clustering groups cells by global transcriptomic identity. Our framework reveals states that are \textbf{\textbf{orthogonal}} to this identity.

% --- FIGURE 4: COMBINED UMAP + STRESS VIOLIN ---
\begin{center}
    \begin{minipage}{0.48\linewidth}
        \includegraphics[width=\linewidth,height=14cm,keepaspectratio]{./images/combined_umap_confetti.png}
    \end{minipage}
    \hfill
    \begin{minipage}{0.48\linewidth}
        \includegraphics[width=\linewidth,height=14cm,keepaspectratio]{./images/Violin_Stress_vs_OtherPD.png}
    \end{minipage}
    \captionof{figure}{\textbf{States are Orthogonal to Identity.} Left: Functional failure modes are scattered across diverse transcriptomic clusters. Right: These states represent a distinct, higher tier of cellular stress compared to the general PD background.}
\end{center}

\columnbreak % FORCE BREAK TO SECOND COLUMN

\section*{6. Molecular \& Regulatory Architecture}
\label{sec:orgc5e7d10}
We identified the specific signaling pathways and protein machinery
assoicated with these states.

\subsection*{The "Regeneration" Machinery (Rule 3)}
\label{sec:org041d8f3}
STRING interaction analysis reveals that Rule 3 markers form a \textbf{\textbf{coordinated structural remodeling network}}, indicating active reorganization of the cell architecture.

% --- FIGURE 5: STRING NETWORK ---
\begin{center}
    \includegraphics[width=0.85\linewidth,height=25cm,keepaspectratio]{./images/string_normal_image.png}
    \captionof{figure}{\textbf{Coordinated Repair Program.} The cytoskeletal regulator \textit{PALD1} is functionally coupled to the matrix remodeler \textit{ADAMTS14}, linking internal actin dynamics with external scaffolding.}
\end{center}

\subsection*{Distinct Regulatory Architectures}
\label{sec:org727b14e}
We explored specific signaling pathways of interest by calculating single-cell module scores. This analysis characterizes the distinct regulatory profile associated with each failure mode.

% --- FIGURE 6: PATHWAY VIOLINS (Pyramid Layout) ---
\begin{center}
    % ROW 1: The Drivers
    \begin{minipage}{0.48\linewidth}
        \includegraphics[width=\linewidth,height=16cm,keepaspectratio]{./images/Violin_dan3_regulon_TCF7L2_targets.png}
        \centerline{\small \textbf{Rule 4: TCF7L2 (Regeneration)}}
    \end{minipage}
    \hfill
    \begin{minipage}{0.48\linewidth}
        \includegraphics[width=\linewidth,height=16cm,keepaspectratio]{./images/Violin_mapk_genes_in_wikipathways_txt.png}
        \centerline{\small \textbf{Rule 2: MAPK (Stress)}}
    \end{minipage}
    
    \vspace{0.5cm} 
    
    % ROW 2: The Consequence
    \begin{minipage}{0.48\linewidth}
        \includegraphics[width=\linewidth,height=16cm,keepaspectratio]{./images/Violin_DopamineKEGGpathway_txt.png}
        \centerline{\small \textbf{Rule 1: Dopamine (Metabolic)}}
    \end{minipage}
    
    \captionof{figure}{\textbf{Distinct Signaling Mechanisms.} \textit{TCF7L2} (Wnt) is upregulated in Rule 4. ($P<0.001$). \textit{MAPK} drives Rule 2 ($P<0.05$). \textit{Dopamine Metabolism} is downregulated in Rule 1 ($P<0.01$).}
\end{center}
\section*{7. Pseudotime Dynamics}
\label{sec:orgccc5cba}
We mapped the subgroups onto a pseudotime disease trajectory.

% --- FIGURE 7: PSEUDOTIME DENSITY ---
\begin{center}
    \includegraphics[width=0.95\linewidth,height=18
    cm,keepaspectratio]{./images/PD_Pseudotime_Chronology.png}
    \captionof{figure}{\textbf{Disease Progression Landscape.} Metabolic dysfunction (Rule 1) presents as a broad baseline signal. The \textbf{Regenerative State (Rule 4)} is sharply localized to advanced pseudotime, followed by the peak density of \textbf{Iron Toxicity (Rule 2)}, suggesting these are distinct late-stage phenomena.}
\end{center}


\section*{Conclusion}
\label{sec:orgc235af1}
We have mapped the regulatory crosstalk of PD into three potentially actionable states:

\begin{enumerate}
\item \textbf{\textbf{Metabolic (Baseline):}} Mitochondrial support.
\item \textbf{\textbf{Iron-Toxic (Acute):}} Marked by MAPK and functional demand.
\item \textbf{\textbf{Regenerative (Structural):}} A coordinated but frustrated attempt at repair linked to TCF7L2.
\end{enumerate}

\end{multicols}
\end{document}
